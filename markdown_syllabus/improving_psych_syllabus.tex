\documentclass[11pt,man]{article}
\usepackage[margin=1in]{geometry}
\newcommand*{\authorfont}{\fontfamily{phv}\selectfont}
\usepackage[]{mathpazo}
\usepackage{abstract}
\renewcommand{\abstractname}{}    % clear the title
\renewcommand{\absnamepos}{empty} % originally center
\newcommand{\blankline}{\quad\pagebreak[2]}

\providecommand{\tightlist}{%
  \setlength{\itemsep}{0pt}\setlength{\parskip}{0pt}} 
\usepackage{longtable,booktabs}

\usepackage{parskip}
\usepackage{titlesec}
\titlespacing\section{0pt}{12pt plus 4pt minus 2pt}{6pt plus 2pt minus 2pt}
\titlespacing\subsection{0pt}{12pt plus 4pt minus 2pt}{6pt plus 2pt minus 2pt}

\titleformat*{\subsubsection}{\normalsize\itshape}

\usepackage{titling}
\setlength{\droptitle}{-.25cm}

%\setlength{\parindent}{0pt}
%\setlength{\parskip}{6pt plus 2pt minus 1pt}
%\setlength{\emergencystretch}{3em}  % prevent overfull lines 

\usepackage[T1]{fontenc}
\usepackage[utf8]{inputenc}

\usepackage{fancyhdr}
\pagestyle{fancy}
\usepackage{lastpage}
\renewcommand{\headrulewidth}{0.3pt}
\renewcommand{\footrulewidth}{0.0pt} 
\lhead{}
\chead{}
\rhead{\footnotesize Improving Research in Psychological Science (Psych 711) -- Spring 2019}
\lfoot{}
\cfoot{\small \thepage/\pageref*{LastPage}}
\rfoot{}

\fancypagestyle{firststyle}
{
\renewcommand{\headrulewidth}{0pt}%
   \fancyhf{}
   \fancyfoot[C]{\small \thepage/\pageref*{LastPage}}
}

%\def\labelitemi{--}
%\usepackage{enumitem}
%\setitemize[0]{leftmargin=25pt}
%\setenumerate[0]{leftmargin=25pt}




\makeatletter
\@ifpackageloaded{hyperref}{}{%
\ifxetex
  \usepackage[setpagesize=false, % page size defined by xetex
              unicode=false, % unicode breaks when used with xetex
              xetex]{hyperref}
\else
  \usepackage[unicode=true]{hyperref}
\fi
}
\@ifpackageloaded{color}{
    \PassOptionsToPackage{usenames,dvipsnames}{color}
}{%
    \usepackage[usenames,dvipsnames]{color}
}
\makeatother
\hypersetup{breaklinks=true,
            bookmarks=true,
            pdfauthor={ ()},
             pdfkeywords = {},  
            pdftitle={Improving Research in Psychological Science (Psych 711)},
            colorlinks=true,
            citecolor=blue,
            urlcolor=blue,
            linkcolor=magenta,
            pdfborder={0 0 0}}
\urlstyle{same}  % don't use monospace font for urls


\setcounter{secnumdepth}{0}





\usepackage{setspace}

\title{Improving Research in Psychological Science (Psych 711)}
\author{Gary Lupyan}
\date{Spring 2019}


\begin{document}  

		\maketitle
		
	
		\thispagestyle{firststyle}

%	\thispagestyle{empty}


	\noindent \begin{tabular*}{\textwidth}{ @{\extracolsep{\fill}} lr @{\extracolsep{\fill}}}


E-mail: \texttt{\href{mailto:lupyan@wisc.edu}{\nolinkurl{lupyan@wisc.edu}}} & Web: \href{http://sapir.psych.wisc.edu}{\tt sapir.psych.wisc.edu}\\
Office Hours: By appointment  &  Class Hours: Weds 9:00am-11:30am\\
Office: Psych 419  & Class Room: Psych 311\\
	&  \\
	\hline
	\end{tabular*}
	
\vspace{2mm}
	


\section{Course Description}\label{course-description}

This course was born from the conviction that graduate students do not
get sufficient time to (1) think about the ``big picture'' of the
project that we as social scientists are engaged in and (2) to have
hands-on experience with scientific and expositional best-practices.
This is especially important now, as decades-old defaults about null
hypothesis testing, statistical power, and expectations concerning open
data, are being overturned.

\section{Course Learning Outcomes}\label{course-learning-outcomes}

This course will:

\begin{enumerate}
\def\labelenumi{\arabic{enumi}.}
\item
  Give you a stronger background in producing empirical research that is
  informed by insights from philosophy of science, statistical truths
  (which are frequently counter-intuitive), and real-world issues that
  researchers encounter when conducting psychological research.
\item
  Provide models for efficient data-management, analysis pipelines,
  data-visualization, and reproducible workflow--practices essential for
  producing high quality research and saving you time and nerves.
\item
  Improve the clarity and precision of your presentations and writing
  skills through workshopping and peer-feedback.
\item
  Professionalize your approach to doing science.
\end{enumerate}

\section{Expectations}\label{expectations}

Students will need to read and critically engage with all the assigned
readings. Look for passages that you find especially challenging,
thought provoking, or controversial. To help you engage with the
readings and to avoid falling behind, you will be expected to post a
Question, Response, or Comment (QRC) for each reading by Monday 8pm
prior to each class. There will also be a number of assignments, as
noted on the schedule below. The assignments are due on Tuesday at 8pm
before class. Detailed instructions about each assignment will be
provided at least 2 weeks before the due date. You will want to do the
readings for the week the assignment is due before starting the
assignment. Please note that some weeks have considerably more reading
than other weeks. Plan your schedule accordingly.

\section{How Credit Hours are Met by the Course (Times are
estimates)}\label{how-credit-hours-are-met-by-the-course-times-are-estimates}

\begin{itemize}
\tightlist
\item
  35 hrs class-time
\item
  40 hrs readings of articles and online resources
\item
  38 hrs assignments
\item
  12 hrs final presentation\\
  = 135 hrs Total
\end{itemize}

\subsection{Grading Policy}\label{grading-policy}

\begin{itemize}
\tightlist
\item
  \textbf{25\%} Attendance / in-class participation / final presentation
\item
  \textbf{40\%} QRCs on the weekly readings
\item
  \textbf{35\%} Assignments
\end{itemize}

\section[Class Schedule]{\texorpdfstring{Class Schedule\footnote{As this
  is the first time this class is being offered, the schedule is subject
  to being revised.}}{Class Schedule}}\label{class-schedule1}

\subsection{Part 1: What are we doing?}\label{part-1-what-are-we-doing}

\subsubsection{Week 01, Wed, 01/23}\label{week-01-wed-0123}

\textbf{Discussion of the scientific method as it pertains to
psychology}

\begin{itemize}
\tightlist
\item
  Science, Cargo Cult Science, and Strong Inference

  \begin{itemize}
  \tightlist
  \item
    Read in-class (Feynman, 1974; Platt, 1964)
  \end{itemize}
\item
  The status quo of Psychological Science
\item
  What makes psychology hard?
\end{itemize}

\subsubsection{Week 02, Wed, 01/30}\label{week-02-wed-0130}

\textbf{Our obsession with significance}

To read before class: Cohen (1994); Meehl (1967); Rozin (2001);
optional: Meehl (1978)

\begin{itemize}
\tightlist
\item
  What's up with null hypothesis testing?
\item
  Thinking deeply about confounds and ``escape clauses''
\item
  How might reliance on null hypothesis testing shape our theoretical
  landscape?
\item
  In-class exercise: Motivating your current research projects

  \begin{itemize}
  \tightlist
  \item
    What makes a question \emph{interesting}? What makes it important?
  \end{itemize}
\end{itemize}

\subsubsection{Week 03, Wed, 02/06}\label{week-03-wed-0206}

\textbf{Ways of doing science: relationships between hypotheses, theory,
methods, and data}

To read before class: Newell (1973) pp.~1-9; Kosslyn (2006); Greenwald
(2012); Schaller (2016); optional Rozin (2006)

\begin{itemize}
\tightlist
\item
  \textbf{Assignment 1 due: Tightening causal inferences}
\item
  What questions do we tend to ask?

  \begin{itemize}
  \tightlist
  \item
    Question-taking vs.~question-making disciplines
  \end{itemize}
\item
  What should a good theory do?
\item
  Prediction vs.~understanding
\item
  Discussion of Assignment 1
\item
  In-class exercise: Motivating your current research projects, a
  reprise.
\end{itemize}

\subsubsection{Week 04, Wed, 02/13}\label{week-04-wed-0213}

\textbf{Identifying some problematic practices in our field.}

To read before class: John, Loewenstein, \& Prelec (2012); Nelson,
Simmons, \& Simonsohn (2018)

\begin{itemize}
\tightlist
\item
  Questionable research practices

  \begin{itemize}
  \tightlist
  \item
    Dance of the p-values (www.youtube.com/watch?v=5OL1RqHrZQ8)
  \item
    Various flavors of HARKing
  \end{itemize}
\item
  Power Analysis (hands-on practice)

  \begin{itemize}
  \tightlist
  \item
    How underpowered is your favorite study?
  \item
    How underpowered is \emph{your} study?
  \end{itemize}
\item
  When should we trust empirical findings?
\end{itemize}

\subsection{Part 2: How can we do it
better?}\label{part-2-how-can-we-do-it-better}

\subsubsection{Week 05, Wed, 02/20}\label{week-05-wed-0220}

\textbf{Psychology 2.0: Why this time is different}

To read before class: Spellman (2015); Munafò et al. (2017); Zwaan, Etz,
Lucas, \& Donnellan (2018) (read target article and any commentaries
that look interesting to you)

\begin{itemize}
\item
  \textbf{Assignment 2 due: Trace back of a prominent replication
  failure}
\item
  Why this time is different
\item
  Replication: A crisis?
\item
  Varieties of replication
\end{itemize}

\newpage

\subsubsection{Week 06, Wed, 02/27}\label{week-06-wed-0227}

\textbf{Avoiding questionable research practices.}

To read before class: Simonsohn (2015); Wicherts et al. (2016); Nosek,
Ebersole, DeHaven, \& Mellor (2018); Simons, Shoda, \& Lindsay (2017);
\url{https://sometimesimwrong.typepad.com/wrong/2015/06/why-p-048-should-be-rare-and-why-this-feels-counterintuitive.html};
play with \url{https://rpsychologist.com/d3/pdist/}

\begin{itemize}
\tightlist
\item
  Telescopes: large and small
\item
  Experimenter degrees of freedom and the garden of forking paths
\item
  Pre-registration
\item
  Transparency
\item
  Discussion of Assignment 2
\end{itemize}

\subsection{Improving your workflow}\label{improving-your-workflow}

\subsubsection{Week 07, Wed, 03/06}\label{week-07-wed-0306}

\textbf{Naming conventions and data wrangling}

To do before class:

If you are shaky on data manipulation in R and/or are unfamiliar with
dplyr/pipes, read through sections 5, 11, and 12 of
\url{https://r4ds.had.co.nz/}

If you know what ``tidy data'' are and are comfortable with data
manipulation in R, walk yourself through
\url{https://cran.r-project.org/web/packages/tidyr/vignettes/tidy-data.html}
to cement these skills.

Inspect the Tidyverse packages:
\url{https://www.tidyverse.org/packages/} ~ Look over ``cheet sheets''
that are useful to you:
\url{https://www.rstudio.com/resources/cheatsheets/} ~

\begin{itemize}
\tightlist
\item
  Exercise: Interpret your partner's dataset and have them interpret
  yours
\item
  How are you currently managing your data? Is it working?
\item
  The importance of naming conventions
\item
  Organizing your data so that you and others can understand it
\item
  Living in the \emph{Tidyverse}

  \begin{itemize}
  \tightlist
  \item
    Exercise: Practice with data-wrangling and tidying
  \end{itemize}
\end{itemize}

\subsubsection{Week 08, Wed, 03/13}\label{week-08-wed-0313}

\textbf{Better reproducibility through technology}

To do before class: Install \textbf{papaja} and read through the
tutorial: \url{https://crsh.github.io/papaja_man/introduction.html};
send to me through

\begin{itemize}
\tightlist
\item
  \textbf{Assignment 3 due}: Wrangle GSS data to answer the posed
  questions
\item
  Reproducible analyses using R markdown
\item
  Using \emph{papaja} to create reproducible documents
\item
  In class exercise: Convert the provided sample dataset into a
  APA-styled results section
\item
  In class exercise: begin Assignment 4: In pairs, reproduce a results
  section from a \emph{Psych Science} paper w/ open materials using R
  markdown/papaja**
\end{itemize}

\subsubsection{Week 10, Wed, 03/27}\label{week-10-wed-0327}

\textbf{Data sharing}

To read before class: Meyer (2018); Gilmore, Kennedy, \& Adolph (2018);
Vuorre \& Curley (2018)

Install git (\url{https://git-scm.com/})

Register for a github account (\url{https://github.com/})

Go through \emph{Learn Git Basics} at
\url{https://backlog.com/git-tutorial/what-is-git/}

\begin{itemize}
\tightlist
\item
  \textbf{Assignment 4 due}
\item
  The importance of sharing
\item
  Why are we not sharing?
\item
  Best practices in data/code sharing
\item
  Version control and collaboration using \emph{git/github} (crash
  course by Pierce Edmiston)
\end{itemize}

\subsection{Improving your data
visualization}\label{improving-your-data-visualization}

\subsubsection{Week 11, Wed, 04/03}\label{week-11-wed-0403}

\textbf{Making visualization work}

To read before class: selected sections of Healy (2018).

\begin{itemize}
\tightlist
\item
  \textbf{Assignment 5 due}: Find two good and two bad scientific
  visualizations
\item
  Data visualizations that work and that do not work
\item
  Using visualization to help understand what you've discovered

  \begin{itemize}
  \tightlist
  \item
    Effective visualization for data-exploration using \emph{(ggplot)}
  \end{itemize}
\item
  Helping \emph{others} understand what you've discovered

  \begin{itemize}
  \tightlist
  \item
    Exercise: Graph these data
  \end{itemize}
\end{itemize}

\subsection{Improving your writing}\label{improving-your-writing}

\subsubsection{Week 12, Wed, 04/10}\label{week-12-wed-0410}

\textbf{It's not done until it's written}

To read before class: \emph{Why academics stink at writing}:
\url{http://stevenpinker.com/why-academics-stink-writing}; Chapters 1-5
of Schimel (2011); Mensh \& Kording (2017); Dan Simons's writing guide:
\url{http://www.dansimons.com/resources/Simons_on_writing_1.4.pdf};
\textbf{Find 2 samples of good and 2 samples of bad scientific writing}

\begin{itemize}
\tightlist
\item
  \textbf{Assignment 6 due: Create visualizations for the listed
  analyses of GSS data}
\item
  Deconstructing good and bad writing
\item
  What makes writing \emph{good} writing?

  \begin{itemize}
  \tightlist
  \item
    Minding the knowledge gap
  \item
    Staying coherent
  \end{itemize}
\end{itemize}

\subsubsection{Week 13, Wed, 04/17}\label{week-13-wed-0417}

\textbf{Writing it better}

\begin{itemize}
\tightlist
\item
  \textbf{Assignment 7 due: Write a research statement}
\item
  Exercise: Peer critiques of each other's writing
\end{itemize}

\subsection{Improving your
presentations}\label{improving-your-presentations}

\subsubsection{Week 14, Wed, 04/24}\label{week-14-wed-0424}

\textbf{Making your presentations more effective.}

To read before class: Talk tips (see Google Drive) * \textbf{Assignment
8 due: Improve a past presentation}

\begin{itemize}
\tightlist
\item
  The narrative arc in an oral presentation
\item
  The knowledge gap revisited
\item
  Some tips on slide design
\end{itemize}

\subsubsection{Week 15, Wed, 05/01}\label{week-15-wed-0501}

\begin{itemize}
\item
  \textbf{Assignment 9 due: 5 min intro to a longer presentation: before
  vs.~after} \textbf{Final presentations}
\item
  Give a practiced 5 minute intro to a longer talk.
\end{itemize}

\subsection{Ethics of Being a Student in the Department of
Psychology:}\label{ethics-of-being-a-student-in-the-department-of-psychology}

The members of the faculty of the Department of Psychology at UW-Madison
uphold the highest ethical standards of teaching and research. They
expect their students to uphold the same standards of ethical conduct.
By registering for this course, you are implicitly agreeing to conduct
yourself with the utmost integrity throughout the semester.

In the Department of Psychology, acts of academic misconduct are taken
very seriously. Such acts diminish the educational experience for all
involved -- students who commit the acts, classmates who would never
consider engaging in such behaviors, and instructors. Academic
misconduct includes, but is not limited to, cheating on assignments and
exams, stealing exams, sabotaging the work of classmates, submitting
fraudulent data, plagiarizing the work of classmates or published and/or
online sources, acquiring previously written papers and submitting them
(altered or unaltered) for course assignments, collaborating with
classmates when such collaboration is not authorized, and assisting
fellow students in acts of misconduct. Students who have knowledge that
classmates have engaged in academic misconduct should report this to the
instructor.

\subsection{Complaints:}\label{complaints}

Occasionally, a student may have a complaint about a TA or course
instructor. If that happens, you should feel free to discuss the matter
directly with the TA or instructor. If the complaint is about the TA and
you do not feel comfortable discussing it with him or her, you should
discuss it with the course instructor. Complaints about mistakes in
grading should be resolved with the TA and/or instructor in the great
majority of cases. If the complaint is about the instructor (other than
ordinary grading questions) and you do not feel comfortable discussing
it with him or her, make an appointment to speak to the Associate Chair
for Undergraduate Studies, Professor Maryellen MacDonald,
\href{mailto:mcmacdonald@wisc.edu}{\nolinkurl{mcmacdonald@wisc.edu}}.

If your complaint concerns sexual harassment, you may also take your
complaint to Dr.~Linnea Burk, Clinical Associate Professor and Director,
Psychology Research and Training Clinic, Room 315 Psychology (262-9079;
\href{mailto:burk@wisc.edu}{\nolinkurl{burk@wisc.edu}}).

If you have concerns about climate or bias in this class, or if you wish
to report an incident of bias or hate that has occurred in class, you
may contact the Chair of the Psychology Department Climate \& Diversity
Committee, Karl Rosengren
(\href{mailto:krosengren@wisc.edu}{\nolinkurl{krosengren@wisc.edu}}).
You may also use the University's bias incident reporting system, which
you can reach at the following link:
\url{https://doso.students.wisc.edu/services/bias-reporting-process/}.

\subsection{Accommodations Policy:}\label{accommodations-policy}

The University of Wisconsin-Madison supports the right of all enrolled
students to a full and equal educational opportunity. The Americans with
Disabilities Act (ADA), Wisconsin State Statute (36.12), and UW-Madison
policy (Faculty Document 1071) require that students with disabilities
be reasonably accommodated in instruction and campus life. Reasonable
accommodations for students with disabilities is a shared faculty and
student responsibility. Students are expected to inform faculty {[}me{]}
of their need for instructional accommodations by the end of the third
week of the semester, or as soon as possible after a disability has been
incurred or recognized. Faculty {[}I{]}, will work either directly with
the student {[}you{]} or in coordination with the McBurney Center to
identify and provide reasonable instructional accommodations. Disability
information, including instructional accommodations, as part of a
student's educational record is confidential and protected under FERPA.

\newpage

\section*{References}\label{references}
\addcontentsline{toc}{section}{References}

\hypertarget{refs}{}
\hypertarget{ref-cohen_earth_1994}{}
Cohen, J. (1994). The earth is round (p\(<\).05). \emph{American
Psychologist}, \emph{49}(12), 997--1003.
\url{https://doi.org/10.1037/0003-066X.49.12.997}

\hypertarget{ref-feynman_cargo_1974-1}{}
Feynman, R. P. (1974). Cargo cult science. \emph{Engineering and
Science}, \emph{37}(7), 10--13.

\hypertarget{ref-gilmore_practical_2018}{}
Gilmore, R. O., Kennedy, J. L., \& Adolph, K. E. (2018). Practical
Solutions for Sharing Data and Materials From Psychological Research.
\emph{Advances in Methods and Practices in Psychological Science},
\emph{1}(1), 121--130. \url{https://doi.org/10.1177/2515245917746500}

\hypertarget{ref-greenwald_there_2012}{}
Greenwald, A. G. (2012). There Is Nothing So Theoretical as a Good
Method. \emph{Perspectives on Psychological Science}, \emph{7}(2),
99--108. \url{https://doi.org/10.1177/1745691611434210}

\hypertarget{ref-healy_data_2018}{}
Healy, K. (2018). \emph{Data Visualization: A Practical Introduction}.
Princeton, NJ: Princeton University Press.

\hypertarget{ref-john_measuring_2012}{}
John, L. K., Loewenstein, G., \& Prelec, D. (2012). Measuring the
Prevalence of Questionable Research Practices With Incentives for Truth
Telling. \emph{Psychological Science}, \emph{23}(5), 524--532.
\url{https://doi.org/10.1177/0956797611430953}

\hypertarget{ref-kosslyn_you_2006}{}
Kosslyn, S. M. (2006). You can play 20 questions with nature and win:
Categorical versus coordinate spatial relations as a case study.
\emph{Neuropsychologia}, \emph{44}(9), 1519--1523.
\url{https://doi.org/10.1016/j.neuropsychologia.2006.01.022}

\hypertarget{ref-meehl_theory-testing_1967}{}
Meehl, P. E. (1967). Theory-Testing in Psychology and Physics: A
Methodological Paradox. \emph{Philosophy of Science}, \emph{34}(2),
103--115. \url{https://doi.org/10.1086/288135}

\hypertarget{ref-meehl_theoretical_1978}{}
Meehl, P. E. (1978). Theoretical risks and tabular asterisks: Sir Karl,
Sir Ronald, and the slow progress of soft psychology. \emph{Journal of
Consulting and Clinical Psychology}, \emph{46}(4), 806--834.
\url{https://doi.org/10.1037/0022-006X.46.4.806}

\hypertarget{ref-mensh_ten_2017}{}
Mensh, B., \& Kording, K. (2017). Ten simple rules for structuring
papers. \emph{PLOS Computational Biology}, \emph{13}(9), e1005619.
\url{https://doi.org/10.1371/journal.pcbi.1005619}

\hypertarget{ref-meyer_practical_2018}{}
Meyer, M. N. (2018). Practical Tips for Ethical Data Sharing.
\emph{Advances in Methods and Practices in Psychological Science},
\emph{1}(1), 131--144. \url{https://doi.org/10.1177/2515245917747656}

\hypertarget{ref-munafo_manifesto_2017}{}
Munafò, M. R., Nosek, B. A., Bishop, D. V. M., Button, K. S., Chambers,
C. D., Percie du Sert, N., \ldots{} Ioannidis, J. P. A. (2017). A
manifesto for reproducible science. \emph{Nature Human Behaviour},
\emph{1}(1), 0021. \url{https://doi.org/10.1038/s41562-016-0021}

\hypertarget{ref-nelson_psychologys_2018}{}
Nelson, L. D., Simmons, J., \& Simonsohn, U. (2018). Psychology's
Renaissance. \emph{Annual Review of Psychology}, \emph{69}(1), 511--534.
\url{https://doi.org/10.1146/annurev-psych-122216-011836}

\hypertarget{ref-newell_you_1973}{}
Newell, A. (1973). You Can't Play 20 Questions with Nature and Win:
Projective Comments on the Papers of This Symposium. In W. Chase (Ed.),
\emph{Visual information processing} (pp. 283--308). New York: Academic
Press.

\hypertarget{ref-nosek_preregistration_2018}{}
Nosek, B. A., Ebersole, C. R., DeHaven, A. C., \& Mellor, D. T. (2018).
The preregistration revolution. \emph{Proceedings of the National
Academy of Sciences}, \emph{115}(11), 2600--2606.
\url{https://doi.org/10.1073/pnas.1708274114}

\hypertarget{ref-platt_strong_1964}{}
Platt, J. R. (1964). Strong Inference: Certain systematic methods of
scientific thinking may produce much more rapid progress than others.
\emph{Science}, \emph{146}(3642), 347--353.
\url{https://doi.org/10.1126/science.146.3642.347}

\hypertarget{ref-rozin_social_2001}{}
Rozin, P. (2001). Social Psychology and Science: Some Lessons From
Solomon Asch. \emph{Personality and Social Psychology Review},
\emph{5}(1), 2--14. \url{https://doi.org/10.1207/S15327957PSPR0501_1}

\hypertarget{ref-rozin_domain_2006}{}
Rozin, P. (2006). Domain Denigration and Process Preference in Academic
Psychology. \emph{Perspectives on Psychological Science}, \emph{1}(4),
365--376. \url{https://doi.org/10.1111/j.1745-6916.2006.00021.x}

\hypertarget{ref-schaller_empirical_2016}{}
Schaller, M. (2016). The empirical benefits of conceptual rigor:
Systematic articulation of conceptual hypotheses can reduce the risk of
non-replicable results (and facilitate novel discoveries too).
\emph{Journal of Experimental Social Psychology}, \emph{66}, 107--115.
\url{https://doi.org/10.1016/j.jesp.2015.09.006}

\hypertarget{ref-schimel_writing_2011}{}
Schimel, J. (2011). \emph{Writing Science: How to Write Papers That Get
Cited and Proposals That Get Funded} (1 edition). Oxford ; New York:
Oxford University Press.

\hypertarget{ref-simons_constraints_2017}{}
Simons, D. J., Shoda, Y., \& Lindsay, D. S. (2017). Constraints on
Generality (COG): A Proposed Addition to All Empirical Papers.
\emph{Perspectives on Psychological Science}, \emph{12}(6), 1123--1128.
\url{https://doi.org/10.1177/1745691617708630}

\hypertarget{ref-simonsohn_small_2015}{}
Simonsohn, U. (2015). Small Telescopes: Detectability and the Evaluation
of Replication Results. \emph{Psychological Science}, \emph{26}(5),
559--569. \url{https://doi.org/10.1177/0956797614567341}

\hypertarget{ref-spellman_short_2015}{}
Spellman, B. A. (2015). A Short (Personal) Future History of Revolution
2.0. \emph{Perspectives on Psychological Science}, \emph{10}(6),
886--899. \url{https://doi.org/10.1177/1745691615609918}

\hypertarget{ref-vuorre_curating_2018}{}
Vuorre, M., \& Curley, J. P. (2018). Curating Research Assets: A
Tutorial on the Git Version Control System. \emph{Advances in Methods
and Practices in Psychological Science}, \emph{1}(2), 219--236.
\url{https://doi.org/10.1177/2515245918754826}

\hypertarget{ref-wicherts_degrees_2016}{}
Wicherts, J. M., Veldkamp, C. L. S., Augusteijn, H. E. M., Bakker, M.,
van Aert, R. C. M., \& van Assen, M. A. L. M. (2016). Degrees of Freedom
in Planning, Running, Analyzing, and Reporting Psychological Studies: A
Checklist to Avoid p-Hacking. \emph{Frontiers in Psychology}, \emph{7}.
\url{https://doi.org/10.3389/fpsyg.2016.01832}

\hypertarget{ref-zwaan_making_2018}{}
Zwaan, R. A., Etz, A., Lucas, R. E., \& Donnellan, M. B. (2018). Making
replication mainstream. \emph{Behavioral and Brain Sciences}, \emph{41}.
\url{https://doi.org/10.1017/S0140525X17001972}




\end{document}

\makeatletter
\def\@maketitle{%
  \newpage
%  \null
%  \vskip 2em%
%  \begin{center}%
  \let \footnote \thanks
    {\fontsize{18}{20}\selectfont\raggedright  \setlength{\parindent}{0pt} \@title \par}%
}
%\fi
\makeatother
